% Copyright(C) 2011-2015 Pedro H. Penna <pedrohenriquepenna@gmail.com>
%
% This file is part of Nanvix.
%
% Nanvix is free software; you can redistribute it and/or modify
% it under the terms of the GNU General Public License as published by
% the Free Software Foundation; either version 3 of the License, or
% (at your option) any later version.
%
% Nanvix is distributed in the hope that it will be useful,
% but WITHOUT ANY WARRANTY; without even the implied warranty of
% MERCHANTABILITY or FITNESS FOR A PARTICULAR PURPOSE. See the
% GNU General Public License for more details.
%
% You should have received a copy of the GNU General Public License
% along with Nanvix. If not, see <http://www.gnu.org/licenses/>.

\documentclass[10pt,a4paper]{article}

% Input.
\usepackage[utf8]{inputenc}
\usepackage[english]{babel}

% Figures.
\usepackage{graphicx}

% References.
\usepackage[backend=biber]{biblatex}

\author{Pedro H. Penna}
\title{The Nanvix Operating System}

\begin{document}

\maketitle

\section{Introduction}
\label{section: introduction}

% About Nanvix.
Nanvix is an operating system created by Pedro H. Penna for educational purposes. It was designed from scratch to be small and simple, and yet modern and fully featured, so that it could help both, novices and experienced enthusiasts in operating systems, to learn about kernel hacking. The first release of Nanvix came out in early 2011, and since then the system has gone through several changes. This paper details the internals of Nanvix 1.2. All previous and future releases are available at \url{github.com/ppenna/nanvix}, under the GPLv3 license.

% Paper organization.
In this section, we present an overview of Nanvix, starting with the system architecture, then presenting the system services, and finally discussing the required hardware to run the system. In later sections, we present a detailed description of each system module.

\subsection{System Architecture}
\label{section: system architecture}

\subsection{System Services}
\label{section: system services}

\subsection{Hardware Requirements}
\label{section: hardware requirements}

\end{document}